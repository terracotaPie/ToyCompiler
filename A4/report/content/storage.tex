\section{Storage}
 We will augment our symbol table to store offsets and lexical levels, as well
 as the PC for each routine to allow for easy branching. We explain in more
 detail below.
\subsection{Variables}
 Each variable stores its offset from 0 and increments the scope's offset, so
 that following variables can calculate their offset. Both \texttt{Boolean}s and
 \texttt{Integer}s will take up one word on the frame. Generally, arrays come in
 the form \texttt{A[len1][len2]}, and so we need to store $len1 \times len2$,
 when both $len1$ and $len2$ are adjusted to be 0 indexed.
\subsubsection{Main}
  When entering a major scope, we will calculate the size required for all
  the variables and allocate space accordingly, and we will give the scope an
  frame/activation record and a block for control information. These variables
  also get the lexical level of the nearest scope. Each child of
  \texttt{SybmolTableObject} will get their lexical level from the
  \texttt{SymbolTable} they are associated with, and their \texttt{offset} is
  based off of 0 from from that SymbolTable. This remains consistent with our
  implementatoin of objects in our \texttt{SymbolTable}. For arrays, we will add
  the initial (possibly negative) bounds so that we have a base to do array
  arithmetic with.  Because of this, we have an array index and a machine array
  index that our classes have to keep track of.

  \begin{code}[Array Indexing]
    A[-1..10] : Integer
    /** The following will be tracked 
      { arrayOffset1 = -1
      , arrayOffset2 = None
      , len1 = 11 
      , len2 = 1
      , ...
      }
    */
    B[5,5..10] : Integer
    /** The following will be tracked 
      { arrayOffset1 = 0
      , arrayOffset2 = 5
      , len1 = 5
      , len2 = 5
      , ...
      }
    */
  \end{code}
  
\subsubsection{Procedures and Functions}
  Because functions and procedures are not the highest scope, they have a
  nonzero offset. Variables declared in procedures and functions will have their
  offset calculated treating the initial offset as 0. So a variable $v$ offset
  $o_v$ from the start of the procedure $p$ at offset $o_p$ has an actual offset
  of $o_v + o_p$.
  
  \begin{code}[Function Offset]
  function hello (a: Integer): Integer
  {
    var x: Bool
    var y: Integer
   
    return a
  }
  \end{code}

  Here $y$ has an offset of 4 words (return + argument + return address + var
  x), but an actual offset of $o_{\text{hello}} + 4$.
  For functions in functions, everything is treated as if it belongs to the top
  function, so the code generation will output the necessary branch.
\subsubsection{Minor Scopes}
  Because minor scopes are independnent (things declared in one minor scope
  cannot be used in another), they may have overlapping space reserved in the
  frame.
\subsection{Integers and Booleans}
  Integers and Booleans are written as is into the assembly code (with
  \texttt{true} as \texttt{MACHINE\_TRUE}, and \texttt{false} as
  \texttt{MACHINE\_FALSE}
\subsection{Text}
  Strings will be pushed character by character starting from the last one.
  \begin{code}[Text]
    // write ``csc488''
    PUSH `8'
    PUSH `8'
    PUSH `4'
    PUSH `c'
    PUSH `s'
    PUSH `c'
    PRINTC
    PRINTC
    PRINTC
    PRINTC
    PRINTC
    PRINTC
  \end{code}
