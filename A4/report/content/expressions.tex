\section{Expression}

Some notation before we start. $<name>$ will refer to an identifier,
$<\#name>$ will be its address, and $<@name>$ will refer to its lexical level.

Code used in the templates that are not given in machine.pdf refer to other code
in this template, where we replace the code with the code in that block. For
example, \texttt{GT} is not defined in machine.pdf, but it is used in some
places and defined explicity in terms of things in machine.pdf. (Basically,
we're saying that this is psuedocode)

\subsection{Constants}
The constants are Text, Integers, and Booleans. As desribed before, these are
directly inserted into code generation (replacing \texttt{true} [\texttt{false}]
with \texttt{MACHINE\_TRUE} [\texttt{MACHINE\_FALSE}]).
Because of the way we insert text constants, we may end up with multiple copies
of the same word, but it's easier to implement.

\subsection{Scalars}
Scalar variables are accessed as follows:

\begin{code}[Accessing scalars]
// x
ADDR <@x> <#x>
LOAD
\end{code}

\subsection{Array elements}
Because we 0-index arrays intenally, arithmetic operations within array brackets
have to be normalized:

\begin{code}[Array Elements and Normalization]
// A[-2] where A[-3..0]
PUSH -2
PUSH 3
ADD
ADDR <@A> <#A>
ADD
LOAD // load A[-2] <=> Machine A[1]

// A[5] where A[3..6]
PUSH 5
PUSH -3
ADD
ADDR <@A> <#A>
ADD
LOAD // load A[5] <=> Machine A[2]

// A[2]  where A[5] (regular)
PUSH 2
PUSH 0
ADD
ADDR <@A> <#A>
ADD
LOAD
\end{code}

\subsection{Arithmetic Operators}
Let $\texttt{op} \in \{\texttt{SUB,ADD,MULT,DIV}\}$. Suppose that we have the
steps to get the values of \texttt{L} and \texttt{R}, then the following is the
template for \texttt{L op R}

\begin{code}[Arithmetic Operators]
generate_code(L)
generate_code(R)
op
\end{code}

\subsection{Comparison Operators}

Let $\texttt{comp} in \{\texttt{LT, GT, EQ, LTE, GTE}\}$. Suppose that we have
steps to get the values of \texttt{L} and \texttt{R}, then the following is the
for \texttt{L comp R}

\begin{code}[Comparison Operators]
generate_code(L)
generate_code(R)
comp
\end{code}


The following are templates for \texttt{GT, GTE, LTE}. (\texttt{EQ} and
\texttt{LT} are available in machine.pdf). These assume that we do these
operations on variables. Replace \texttt{ADDR <@name> <\#name>; LOAD} with 
\texttt{generate\_code(name)} if these are not variables.


\begin{code}[GT]
/* a > b */
PUSH -1
ADDR <@a> <#a>
LOAD
MULT            // -a
PUSH -1
ADDR <@b> <#b>
LOAD
MULT            // -b
LT              // -a < -b <=> b > a
\end{code}

\begin{code}[LTE]
/* a <= b equiv a < b+1 */
ADDR <@a> <#a>
LOAD
PUSH 1
ADDR <@b> <#b>
LOAD
ADD
LT
\end{code}

\begin{code}[GTE]
/* a >= b equiv a+1 > b
PUSH 1
ADDR <@a> <#a>
LOAD
ADD
ADDR <@b> <#b>
LOAD
GT
\end{code}

\subsection{Boolean Operators}
Let \texttt{bop} $\in$ \texttt{AND, OR}, then the following are implementations
for the two in the set and for \texttt{NOT}

\begin{code}[Boolean Operators]
// L bop R 
generate_code(L)
generate_code(R)
bop

// NOT A
generate_code(A)
NOT
\end{code}

Templates for \texttt{AND} and \texttt{NOT}

\begin{code}[NOT]
// NOT a
ADDR <@a> <#a>
LOAD
PUSH MACHINE_FALSE
EQ 
/* 
  a=true, then true=false => false
  a=false, then false=false => true
*/
\end{code}

\begin{code}[AND]
// a&b equiv !a|!b
ADDR <@a> <#a> // 
LOAD           // a 
NOT            // !a
ADDR <@b> <#b> //
NOT            // !b
OR             // !a|!b
\end{code}
\subsection{Conditionals}
Conditionals and If-then-else statements have the same structure. See the
Statements section.
