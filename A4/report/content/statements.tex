\section{Statements}
\subsection{Assignment}

We load the address at which we wish to store the value and then push the value
we want to assign to the stack. If the value is the result of some operation, 
(e.g.: a := 1 + 2)we perform those now to arrive at the result. Thus the top two 
elements of the stack will be the value and address, so we use STORE to assign.

\begin{code}[Assignment]
// x := 1
ADDR <@x> <#x>
PUSH 1
STORE
\end{code}

\subsection{If}

We generate the machine code for the predicate and both true and false blocks.
After evaluating the condition, we branch to the appropriate block (e.g.: if true,
BF will not branch, we execute the true block, then branch to the end).

\begin{code}[IfStatement]
codegen(condition)      // result will be top of stack after this
PUSH <@else>
BF   
codegen(true_block)             
PUSH <@end_line> 
BR           
codegen(false_block)    // else_line
...                     // end_line
\end{code}

\subsection{While and Repeat}
\subsection{Returns}
\subsection{Reading and Writing}
