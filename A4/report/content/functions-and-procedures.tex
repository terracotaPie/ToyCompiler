\section{Functions and Procedures}
\subsection{Activation Record}
\subsection{Entrance Code}

\begin{code}[Entrance Code]
ADDR 0 F
PUSHMT
PUSH Fbody // label to body of function
\end{code}
\subsection{Exit Code}

\begin{code}[Exit Code]
/* 
Swap and pop to preserve the return value while 
popping all the variables in the stack
*/
SWAP
POP
...
/* 
Put return address on top of stack and goto 
the return address 
*/
SWAP
BR
\end{code}

\subsection{Parameter passing}
If a procedure or a function takes $n$ parameters, we move the stack pointer
back $n$ to grab those parameters before setting the display data. We don't have
to store \texttt{PrevD} or \texttt{n} because it is calculated at compile time.

\begin{code}[Parameter Passing]
PUSHMT
PUSH <n>
SUB
SETD <prevD + 1>
\end{code}

\subsection{Function Calls and Value Return}

\begin{code}[Function Calls and Value Return]
PUSH RETURN_ADDRESS
PUSH PARAMS // label for parameters
PUSH FUNC_BODY // label for function body
/* FUNC_BODY handles exit code */  
BR // return value is on top of the stack after the function.
\end{code}

\subsection{Procedure Calls}
This is the same as the function call but there is no return value on top of the
stack.

\subsection{Display Management}
When we enter a new scope, we increment its display. This is handled at compile
time.
